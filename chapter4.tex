\chapter{CURRENT STATUS AND FUTURE PLANS}

\section{CURRENT STATUS}

The primary focus of this research has been on sparse estimation of nonlinear time series models under the Bayesian framework.  Alternative routes led to different types of regime-switching models being applied to different sets of data. In each chapter, there is a clear application stimulating the discussion in this area. 

In Chapter \ref{chap:temp}, we demonstrate the idea of regime-specific model selection. Using an LSTAR model, we show through a chain of simulation studies that Bayesian shrinkage methods offer a flexible solution to parameter selection. Also, we demonstrate a novel treatment of the threshold variable through the use of a dirichlet prior. Rather than defining the threshold variable by a specific lag, we define it by a weighted average distributed across recently known information. Daily maximum water temperature forecasting was the selected application for this research. The research of this paper was submitted to the journal \textit{Communications in Statistics: Simulation and Computation} on June 19, 2017. A final decision has not yet been made.

In Chapter \ref{chap:traffic}, we go beyond Bayesian shrinkage to incorporate model selection. The model selection methodology is focused on selecting the best short-term forecasting model for traffic occupancy 3, 9, and 15 minutes ahead. Traffic data in urban networks is classically hard to model, especially during congested states. Practitioners in this area desire simple interpretable models that can quickly forecast. Combining spatio-temporal information with a nonlinear TAR model allows us to identify traffic regimes and investigate regime specific dynamics. Final reduced models are the product of a process focused on model selection targeting a specific horizon. We apply these methods to relatively small data sets and compare out-of-sample forecasts to naive and seasonal models using relative forecasting metrics. Due to the applied nature of this paper and focus on multistep ahead forecasts,  this research will be submitted to the \textit{International Journal of Forecasting} before the end of the year. 

\section{FUTURE PLANS}

Current results pertaining to the selected traffic data are not yet summarized. Initial results lead us to believe that random walk models are highly competitive compared to more complex alternatives. Based off research in seasonally cointegrated data, error correction models must be considered. The next modeling approach to traffic occupancy will include lagged deviations from seasonal profiles with lagged deviations from short-term ARDL models. The combination of shrinkage and selection can be applied to this new combined model. 

Currently, the full urban network and other traffic variables in the Athen's dataset are not being used. The methods outlined in the working paper can be easily applied to forecast traffic occupancy, volume, and speed at all locations. Furthermore, the utilization of traffic volume and speed in linear and nonlinear models for traffic occupancy could possibly lead to better forecasting. 

With traffic occupancy, we start by applying a logit transformation to model the variable indirectly. Our early approaches to traffic occupancy involved Bayesian beta regression. Simulated proportional time series  illustrated the value of shrinkage estimation in this generalized linear model. Because of the regime-switching behavior seen in traffic flow, model shrinkage and selection methods need to be explored. Logit and log link functions for the mean and variance, respectively, can be modeled using nonlinear predictors. Generalized nonlinear models (GNLMs) can exhibit regime-switching behavior, and optimal Bayesian estimation of these types of models has value for proportional time series.

A phenomenon witnessed in posterior comparisons of linear and nonlinear models breeds interest in forecast combination schemes. Overall forecasting accuracy for both models may be equivalent from a full day of data, while regime-specific forecasting accuracy may identify a clear favorite. Combination schemes of linear, nonlinear, and even naive approaches involving static or dynamic weights could lead to better forecasting than each individual model. 
