\begin{abstract}
The primary objective in time series analysis is forecasting. Raw data often exhibits nonstationary behavior: trends, seasonal cycles, and heteroskedasticity. After data is transformed to a weakly stationary process,  autoregressive moving average (ARMA) models may capture the remaining temporal dynamics to improve forecasting. Estimation of ARMA can be performed through regressing current values on previous realizations and proxy innovations. The classic paradigm fails when dynamics are nonlinear; in this case, parametric, regime-switching specifications model changes in level, ARMA dynamics, and volatility, using a finite number of latent states. If the states can be identified using past endogenous or exogenous information, a threshold autoregressive (TAR) or logistic smooth transition autoregressive (LSTAR) model may simplify complex nonlinear associations to conditional weakly stationary processes. For ARMA, TAR, and STAR, order parameters quantify the extent past information is associated with the future. Unfortunately, even if model orders are known \textit{a priori}, the possibility of over-fitting can lead to sub-optimal forecasting performance. By intentionally overestimating these orders, a linear representation of the full model is exploited   and Bayesian regularization can be used to achieve sparsity. Global-local shrinkage priors for AR, MA, and exogenous coefficients are adopted to pull posterior means toward 0 without over-shrinking relevant effects. This dissertation introduces, evaluates, and compares Bayesian techniques that automatically perform model selection and coefficient estimation of ARMA, TAR, and STAR models. Multiple Monte Carlo experiments illustrate the accuracy of these methods in finding the "true" data generating process. Practical applications demonstrate their efficacy in forecasting.

\end{abstract}
